\documentclass[12pt,a4paper]{article}

% -----------------------------
% Pacotes de configuração
% -----------------------------
\usepackage[utf8]{inputenc}   
\usepackage[T1]{fontenc}      
\usepackage[english,brazilian]{babel} 
\usepackage[left=3cm, right=2cm, top=3cm, bottom=2cm]{geometry} 
\usepackage{setspace}         
\usepackage{indentfirst}      
\usepackage{microtype}        
\usepackage{graphicx}         % Para incluir gráficos e figuras
\usepackage{float}            % Para posicionamento de figuras
\usepackage{hyperref}         % Links clicáveis no PDF

\begin{document}

% ----------------------------------------------------------
% FOLHA DE ROSTO
% ----------------------------------------------------------
\begin{titlepage}
    \begin{center}
        \vspace*{4cm}
        \textbf{\MakeUppercase{Asma no Brasil: o que revelam os dados da PNS 2019 sobre alimentação, estilo de vida e acesso à saúde}}
        
        \vspace{3cm}
    \end{center}
    
    \begin{flushright}
        Isaque Gomes Azevedo \\
        \texttt{isaquegomes623@gmail.com}
    \end{flushright}
    
    \vfill
    
    \begin{center}
        Belo Horizonte \\
        2025
    \end{center}
\end{titlepage}

\pagestyle{plain} % Numeração a partir daqui

% ----------------------------------------------------------
% SUMÁRIO
% ----------------------------------------------------------
\newpage
\tableofcontents
\newpage

% ----------------------------------------------------------
% RESUMO
% ----------------------------------------------------------
\section*{Resumo}
\addcontentsline{toc}{section}{Resumo}
\begin{singlespace}
\noindent
\textbf{Contexto:} A asma é uma das doenças respiratórias crônicas mais prevalentes no Brasil, afetando milhões de pessoas e impactando a qualidade de vida, produtividade e utilização dos serviços de saúde. Fatores relacionados à alimentação, estilo de vida, determinantes sociais e acesso à saúde influenciam sua ocorrência e manejo. 

\textbf{Objetivo:} Descrever o perfil de adultos brasileiros com diagnóstico médico autorreferido de asma na Pesquisa Nacional de Saúde (PNS) 2019 e analisar associações com alimentação, estilo de vida, determinantes sociais e acesso a serviços de saúde. 

\textbf{Métodos:} Estudo transversal com dados da PNS 2019, incluindo todos os indivíduos. O desfecho foi o autorrelato de diagnóstico médico de asma. Variáveis independentes incluíram marcadores de alimentação (consumo de frutas, legumes, ultraprocessados, carne vermelha), estilo de vida (tabagismo, consumo de álcool, prática de atividade física), determinantes sociais (sexo, escolaridade, renda, região, raça/cor) e acesso a serviços de saúde (consultas médicas, plano de saúde, uso do SUS). Foram conduzidas análises descritivas e modelos de regressão logística (brutos e ajustados), considerando o desenho amostral complexo e pesos da pesquisa. 

\textbf{Discussão:} Os achados serão interpretados à luz das desigualdades sociais em saúde e da literatura sobre fatores de risco e proteção para doenças respiratórias. Por ser estudo transversal, os resultados descrevem associações e perfis, sem permitir inferências de causalidade. 

\textbf{Implicações:} O estudo pode subsidiar políticas públicas de prevenção e promoção da saúde respiratória, com foco em grupos vulneráveis.
\vspace{\baselineskip}

\noindent
\textbf{Palavras-chave:} Asma; alimentação; estilo de vida; determinantes sociais da saúde; acesso a serviços de saúde.
\end{singlespace}

\newpage

% ----------------------------------------------------------
% INTRODUÇÃO
% ----------------------------------------------------------
\section{Introdução}
A asma é uma das doenças respiratórias crônicas mais comuns no Brasil e no mundo, caracterizada por inflamação crônica das vias aéreas e episódios recorrentes de dispneia, chiado e tosse. Sua prevalência e impacto estão associados a fatores de risco modificáveis, como hábitos alimentares inadequados, exposição ao tabagismo passivo e ativo, poluição ambiental, além de determinantes sociais e desigualdades no acesso a serviços de saúde. A análise do perfil de indivíduos com asma é fundamental para orientar políticas públicas de prevenção e manejo da doença.

% ----------------------------------------------------------
% METODOLOGIA
% ----------------------------------------------------------
\section{Metodologia}
\subsection{Base de dados}
Os dados foram obtidos da Pesquisa Nacional de Saúde (PNS) 2019, que inclui informações sobre saúde, estilo de vida, alimentação, acesso a serviços de saúde e condições socioeconômicas da população brasileira.

\subsection{População do estudo}
Incluiu todos os indivíduos com diagnóstico médico autorreferido de asma, sem restrição de faixa etária.

\subsection{Variáveis}
\begin{itemize}
    \item \textbf{Desfecho:} Diagnóstico médico de asma (autorreferido).
    \item \textbf{Variáveis independentes:} Alimentação (frutas, legumes, carne vermelha, ultraprocessados), estilo de vida (tabagismo, consumo de álcool, atividade física), determinantes sociais (sexo, escolaridade, renda, região, raça/cor) e acesso a serviços de saúde (consultas médicas, plano de saúde, uso do SUS).
\end{itemize}

\subsection{Análise estatística}
Foram realizadas análises descritivas por UF e região, com frequências absoluta e relativa, além de gráficos. Modelos de regressão logística foram estimados para avaliar associações brutas e ajustadas, considerando o desenho amostral complexo e pesos da pesquisa.

% ----------------------------------------------------------
% RESULTADOS
% ----------------------------------------------------------
\section{Resultados}
\subsection{Diagnóstico por Unidade da Federação (UF)}
\begin{figure}[H]
    \centering
    \includegraphics[width=0.8\textwidth]{grafico_uf.png} % Substitua pelo arquivo gerado
    \caption{Distribuição de casos de asma por UF (absoluto e relativo).}
    \label{fig:uf}
\end{figure}

\subsection{Diagnóstico por Região}
\begin{figure}[H]
    \centering
    \includegraphics[width=0.6\textwidth]{grafico_regiao.png} % Substitua pelo arquivo gerado
    \caption{Distribuição de casos de asma por região (absoluto e relativo).}
    \label{fig:regiao}
\end{figure}

\subsection{Outras análises descritivas}
Podem ser incluídas tabelas de frequência por sexo, escolaridade, renda, hábitos alimentares, tabagismo, atividade física, entre outras.

% ----------------------------------------------------------
% DISCUSSÃO
% ----------------------------------------------------------
\section{Discussão}
Os resultados apontam desigualdades regionais e sociais na prevalência de asma no Brasil. Evidenciam a influência de fatores de estilo de vida, condições socioeconômicas e acesso a serviços de saúde no manejo da doença. A literatura destaca ainda a importância da alimentação saudável como fator protetor em doenças respiratórias. Ressalta-se a limitação da natureza transversal do estudo, que não permite estabelecer causalidade.

% ----------------------------------------------------------
% CONCLUSÃO
% ----------------------------------------------------------
\section{Conclusão}
Este estudo descreve o perfil epidemiológico da asma no Brasil utilizando dados da PNS 2019. Os achados reforçam a necessidade de estratégias de saúde pública voltadas para prevenção, diagnóstico precoce e manejo da doença, com atenção especial a grupos vulneráveis.

% ----------------------------------------------------------
% REFERÊNCIAS
% ----------------------------------------------------------
\section{Referências}
\begin{itemize}
    \item Ministério da Saúde. Pesquisa Nacional de Saúde 2019. Disponível em: \url{https://www.ibge.gov.br/estatisticas/sociais/saude/21892-pns.html}
    \item Global Initiative for Asthma (GINA). Global Strategy for Asthma Management and Prevention. Disponível em: \url{https://ginasthma.org/}
    \item World Health Organization. Asthma. Disponível em: \url{https://www.who.int/news-room/fact-sheets/detail/asthma}
\end{itemize}

\end{document}
