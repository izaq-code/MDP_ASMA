\documentclass[12pt,a4paper]{article}

% -----------------------------
% Pacotes
% -----------------------------
\usepackage[utf8]{inputenc}
\usepackage[T1]{fontenc}
\usepackage[english,brazilian]{babel}
\usepackage[left=3cm, right=2cm, top=3cm, bottom=2cm]{geometry}
\usepackage{setspace}
\usepackage{indentfirst}
\usepackage{microtype}
\usepackage{graphicx}
\usepackage{float}
\usepackage{hyperref}
\usepackage{enumitem}
\usepackage{longtable}
\hypersetup{
  colorlinks=true,
  urlcolor=blue,
  citecolor=black,
  linkcolor=black,
  pdfauthor={Isaque Gomes Azevedo},
  pdftitle={Asma no Brasil: o que revelam os dados da PNS 2019}
}

\begin{document}

% -----------------------------
% Folha de rosto
% -----------------------------
\begin{titlepage}
    \begin{center}
        \vspace*{4cm}
        \textbf{\MakeUppercase{Asma no Brasil: o que revelam os dados da PNS 2019 sobre alimentação, estilo de vida e acesso à saúde}}
        \vspace{3cm}
    \end{center}

    \begin{flushright}
        Isaque Gomes Azevedo \\
        \texttt{isaquegomes623@gmail.com}
    \end{flushright}

    \vfill
    \begin{center}
        Belo Horizonte \\
        2025
    \end{center}
\end{titlepage}

\pagestyle{plain}

% -----------------------------
% Sumário
% -----------------------------
\newpage
\tableofcontents
\newpage

% -----------------------------
% Resumo
% -----------------------------
\section*{Resumo}
\addcontentsline{toc}{section}{Resumo}
\begin{singlespace}
\noindent
\textbf{Contexto:} A asma é uma das doenças respiratórias crônicas mais prevalentes no Brasil, com impacto relevante em qualidade de vida, produtividade e uso do sistema de saúde. Evidências sugerem que alimentação, fatores comportamentais, determinantes sociais e acesso aos serviços modulam tanto a ocorrência quanto o controle da doença.

\textbf{Objetivo:} Estimar a prevalência autorreferida de diagnóstico médico de asma na PNS 2019 e investigar associações com padrões de alimentação, estilo de vida, determinantes sociais e acesso/uso de serviços, identificando grupos de maior vulnerabilidade e potenciais gradientes sociais.

\textbf{Métodos:} Inquérito transversal com dados públicos da PNS 2019 (IBGE). População: adultos residentes em domicílios particulares. Desfecho: diagnóstico médico de asma (autorreferido). Exposições: marcadores de alimentação, comportamentos, determinantes sociais e indicadores de acesso/uso. Análises com ponderação amostral e desenho complexo, estimando prevalências e IC95\%. Modelagem: regressão logística survey-ajustada, marginal effects, checagem de multicolinearidade, seleção hierarquizada. Análises de sensibilidade: exclusão de indivíduos com DPOC/bronquite/enfisema, recorte por faixa etária e sexo, redefinição de marcadores alimentares.

\textbf{Resultados:} Estimativas nacionais e por macrorregião/UF, distribuição segundo sexo, idade e raça/cor, gradientes por escolaridade e renda, diferenças segundo marcadores de dieta, tabagismo e atividade física, associações ajustadas (ORs) com IC95\%, probabilidades preditas e mapas/gráficos regionais. 

\textbf{Discussão:} Achados interpretados à luz das desigualdades sociais em saúde e da literatura sobre fatores de risco/proteção, destacando implicações para o manejo e controle da asma.

\textbf{Implicações:} Subsidiar políticas de prevenção e promoção, qualificar o cuidado na APS/SUS, orientar recursos para grupos/populações mais afetados.

\vspace{\baselineskip}
\noindent
\textbf{Palavras-chave:} Asma; alimentação; estilo de vida; determinantes sociais; acesso a serviços de saúde; PNS 2019.
\end{singlespace}

\newpage

% -----------------------------
% Introdução
% -----------------------------
\section{Introdução}
A asma é uma doença inflamatória crônica das vias aéreas, com exacerbações recorrentes de dispneia, sibilos e tosse, associadas à hiperresponsividade brônquica. Além da suscetibilidade genética e ambiental, fatores modificáveis como tabagismo, dieta de baixa qualidade e sedentarismo, bem como determinantes sociais e barreiras de acesso, influenciam tanto o risco quanto o controle da doença. No Brasil, inquéritos nacionais permitem monitorar o perfil epidemiológico e as desigualdades regionais, oferecendo subsídios para intervenções focalizadas.

% -----------------------------
% Metodologia
% -----------------------------
\section{Metodologia}

\subsection{Desenho e base de dados}
Estudo transversal baseado na Pesquisa Nacional de Saúde (PNS) 2019, conduzida pelo IBGE em parceria com o Ministério da Saúde, com amostra probabilística e representatividade nacional para a população adulta. Microdados públicos: \href{https://biblioteca.ibge.gov.br/visualizacao/livros/liv101764.pdf}{IBGE (PNS 2019) – Relatório Metodológico}.

\subsection{População do estudo}
Incluiu todos os indivíduos com diagnóstico médico autorreferido de asma.

\subsection{Variáveis e definições}
\begin{itemize}[noitemsep,topsep=0pt]
    \item \textbf{Desfecho:} diagnóstico médico de asma (autorreferido).
    \item \textbf{Exposições principais:}
    \begin{itemize}[noitemsep]
        \item \emph{Alimentação}: consumo de frutas/legumes, carne vermelha, bebidas açucaradas e ultraprocessados.
        \item \emph{Estilo de vida}: tabagismo (atual/ex), consumo de álcool, atividade física no lazer.
        \item \emph{Determinantes sociais}: sexo, idade, raça/cor, escolaridade, renda domiciliar per capita, ocupação, macrorregião, zona (urbana/rural).
        \item \emph{Acesso/uso de serviços}: plano de saúde, consulta médica nos últimos 12 meses, uso do SUS, procura por atendimento.
    \end{itemize}
\end{itemize}

\subsection{Análise estatística}
\begin{enumerate}[noitemsep]
    \item Estatística descritiva ponderada e IC95\%.
    \item Regressão logística survey-ajustada, com seleção hierárquica de blocos.
    \item Efeitos marginais e probabilidades preditas.
    \item Sensibilidade: exclusão de DPOC/bronquite/enfisema, recorte por idade/sexo, redefinição de marcadores alimentares.
    \item Diagnósticos: multicolinearidade (VIF), pseudo-$R^2$, \emph{goodness-of-fit}.
\end{enumerate}

\subsection{Aspectos éticos}
A PNS possui aprovação ética nacional; microdados públicos e anonimizados.

% -----------------------------
% Resultados
% -----------------------------
\section{Resultados}

\subsection{Panorama nacional e por UF}
\begin{figure}[H]
    \centering
    \includegraphics[width=0.85\textwidth]{diagnostico_por_uf_regiao-1.pdf}
    \includegraphics[width=0.85\textwidth]{diagnostico_por_uf_regiao-2.pdf}
    \caption{Prevalência autorreferida de asma por UF (absoluto e relativo, IC95\%).}
    \label{fig:uf}
\end{figure}

\subsection{Diferenças regionais}
\begin{figure}[H]
    \centering
    \includegraphics[width=0.85\textwidth]{diagnostico_por_uf_regiao-3.pdf}
    \includegraphics[width=0.85\textwidth]{diagnostico_por_uf_regiao-4.pdf}
    \caption{Distribuição por macrorregião (absoluto e relativo).}
    \label{fig:regioes}
\end{figure}

\subsection{População saudável vs. asmática}
\begin{figure}[H]
    \centering
    \includegraphics[width=0.8\textwidth]{saudaveis_vs_asma_15_65.png}
    \caption{Comparação entre indivíduos saudáveis e doentes (asma) na faixa etária de 15 a 65 anos – PNS 2019.}
    \label{fig:saudaveis_asma}
\end{figure}

A Figura~\ref{fig:saudaveis_asma} mostra que a população saudável na faixa etária de 15 a 65 anos é majoritária em comparação com os indivíduos que referiram diagnóstico de asma. Foram identificados aproximadamente \textbf{76 mil saudáveis} contra apenas \textbf{cerca de 4 mil doentes}, evidenciando a baixa prevalência relativa da doença nesse grupo etário.

\subsection{Distribuição etária dos asmáticos}
\begin{figure}[H]
    \centering
    \includegraphics[width=0.95\textwidth]{idade_asma.png}
    \caption{Distribuição da idade entre os indivíduos com diagnóstico de asma – PNS 2019.}
    \label{fig:idade_asma}
\end{figure}

A Figura~\ref{fig:idade_asma} apresenta a distribuição da idade das pessoas com asma. Observa-se uma maior concentração de casos entre os 25 e 45 anos, com redução gradual após os 60 anos.

Com base nesses dados, calculamos as seguintes estatísticas descritivas da amostra:
\begin{itemize}
    \item \textbf{Média}: aproximadamente \textbf{40 anos}, refletindo a tendência central da população asmática.
    \item \textbf{Mediana}: cerca de \textbf{38 anos}, indicando que metade dos indivíduos asmáticos tem menos de 38 anos.
    \item \textbf{Moda}: situada no intervalo dos \textbf{30 a 35 anos}, faixa com maior frequência de asmáticos.
\end{itemize}

A análise será restrita à população entre \textbf{15 e 65 anos}, eliminando os extremos etários (crianças e idosos acima de 65), de modo a reduzir vieses e garantir maior homogeneidade estatística.

\subsection{Modelo conceitual}
\begin{figure}[H]
    \centering
    \includegraphics[width=0.95\textwidth]{Group 2.png}
    \caption{Mapa conceitual das dimensões, aspectos e atributos relacionados à asma na PNS 2019.}
    \label{fig:mapa}
\end{figure}

\subsection{Tabela de Dimensões, Aspectos e Atributos}
\renewcommand{\arraystretch}{1.4}
\setlength{\tabcolsep}{6pt}
\begin{longtable}{|p{3.8cm}|p{5.2cm}|p{5.2cm}|}
\hline
\textbf{Dimensão} & \textbf{Aspectos} & \textbf{Atributos (variáveis da PNS)} \\
\hline
\endfirsthead
\hline
\textbf{Dimensão} & \textbf{Aspectos} & \textbf{Atributos (variáveis da PNS)} \\
\hline
\endhead

\textbf{Acesso/uso de serviços de saúde} & 
Procura por atendimento, plano de saúde, uso do SUS, consultas médicas & 
Frequência de consultas nos últimos 12 meses; presença de plano de saúde privado; se procurou atendimento nos últimos 2 semanas; se foi atendido; local de atendimento (SUS ou privado) \\
\hline

\textbf{Estilo de vida} & 
Tabagismo, consumo de álcool, prática de atividade física, sono & 
Fumante atual, ex-fumante, idade de início do tabagismo, consumo de álcool (frequência e quantidade), minutos de atividade física no lazer/trabalho, qualidade do sono \\
\hline

\textbf{Alimentação} & 
Consumo de frutas, legumes, carne vermelha, alimentos ultraprocessados, bebidas açucaradas, amamentação & 
Porções diárias de frutas/legumes, frequência semanal de carne vermelha e embutidos, consumo de refrigerantes e sucos industrializados, histórico de aleitamento materno exclusivo (0–6 meses) e complementar (6–12 meses) \\
\hline

\textbf{Determinantes sociais} & 
Sexo, idade, raça/cor, escolaridade, renda, zona geográfica & 
Sexo biológico, faixa etária, raça/cor autodeclarada, anos de estudo, renda domiciliar per capita, macrorregião, situação do domicílio (urbano/rural), ocupação principal \\
\hline

\textbf{Condições clínicas} & 
Diagnóstico médico prévio de doenças respiratórias e comorbidades & 
Presença de asma, DPOC, bronquite crônica, enfisema, outras doenças crônicas (hipertensão, diabetes, etc.) \\
\hline

\textbf{Fatores ambientais} & 
Exposição à poluição, ambientes livres de fumo, sazonalidade climática & 
Relato de poluição no domicílio ou vizinhança, presença de fumantes no domicílio, existência de ambientes 100\% livres de fumo, percepção de variação climática que afeta sintomas \\
\hline

\textbf{Genética/Predisposição} & 
Histórico familiar & 
Histórico de asma ou doenças respiratórias em familiares de primeiro grau \\
\hline
\end{longtable}



% -----------------------------
% Discussão
% -----------------------------
\section{Discussão}
Os achados reforçam desigualdades regionais e sociais na asma, consistentes com a literatura sobre determinantes sociais, papel do tabagismo, sono e dieta. A natureza transversal limita inferência causal; entretanto, a magnitude e consistência dos gradientes sugerem alvos plausíveis para intervenção.

% -----------------------------
% Conclusão
% -----------------------------
\section{Conclusão}
A PNS 2019 permite delinear o perfil nacional da asma e revelar \emph{quem} mais necessita de políticas de prevenção, diagnóstico e manejo. Priorizar promoção da saúde, reduzir barreiras de acesso e qualificar o cuidado longitudinal podem reduzir desigualdades e melhorar o controle da doença.

% -----------------------------
% Implicações para políticas públicas
% -----------------------------
\section{Implicações para políticas públicas}
\begin{itemize}[noitemsep]
  \item Fortalecer APS/SUS com protocolos de manejo da asma.
  \item Foco em grupos vulneráveis: baixa renda/escolaridade, pretos/pardos, regiões com maior prevalência.
  \item Ambientes escolares e de trabalho livres de fumo e poluentes.
  \item Monitoramento contínuo e integração com dados administrativos.
\end{itemize}


% ----------------------------------------------------------
% REVISÃO DA LITERATURA (expandida com links)
% ----------------------------------------------------------
\section{Revisão da Literatura}

\subsection{Estanislau et al. (2021) -- Jornal de Pediatria}
\begin{itemize}[label={}, leftmargin=0cm, itemindent=2cm, align=left]
  \item[\textbf{Objetivo:}] Examinar a associação entre horas de sono e asma atual em adolescentes brasileiros (ERICA).
  \item[\textbf{Metodologia:}] Inquérito escolar nacional (2013--2014; $n\approx59{,}4$ mil), modelos de regressão Poisson com variância robusta, ajuste para múltiplos confundidores.
  \item[\textbf{Resultados:}] Maior prevalência de asma em quem dormia $<7$h/noite; PR ajustada $\approx$1{,}17 (IC95\% 1{,}01--1{,}35).
  \item[\textbf{Principal contribuição:}] Evidencia o sono curto como fator associado à asma na adolescência, sugerindo \emph{sleep hygiene} como componente de manejo.
  \item[\textbf{Link:}] \href{https://pubmed.ncbi.nlm.nih.gov/32956628/}{PubMed} \,|\, \href{https://doi.org/10.1016/j.jped.2020.07.007}{DOI}
\end{itemize}

\subsection{Finkelstein \& Jeong (2017) -- Ann. NY Acad. Sci.}
\begin{itemize}[label={}, leftmargin=0cm, itemindent=2cm, align=left]
  \item[\textbf{Objetivo:}] Personalizar predição precoce de exacerbações de asma com \emph{machine learning}.
  \item[\textbf{Metodologia:}] Modelos supervisionados em dados clínicos/monitoramento; comparação entre algoritmos.
  \item[\textbf{Resultados:}] Desempenho promissor (AUCs elevadas em validações internas), viabilizando alertas antecipados.
  \item[\textbf{Principal contribuição:}] Abre caminho para manejo proativo e intervenções personalizadas baseadas em risco.
  \item[\textbf{Link:}] \href{https://pubmed.ncbi.nlm.nih.gov/27627195}{PubMed} 
\end{itemize}

\subsection{Gonçalves, França \& Zarate (2024) -- Brazilian e-Science Workshop}
\begin{itemize}[label={}, leftmargin=0cm, itemindent=2cm, align=left]
  \item[\textbf{Objetivo:}] Argumentar sobre a centralidade do \emph{conhecimento de domínio} na construção de modelos.
  \item[\textbf{Metodologia:}] Ensaio técnico com exemplos aplicados.
  \item[\textbf{Resultados:}] Falhas de modelagem decorrem da desconexão com o domínio; colaboração interdisciplinar melhora validade.
  \item[\textbf{Principal contribuição:}] Arcabouço para integrar especialistas e cientistas de dados em saúde.
  \item[\textbf{Link:}] \href{https://www.google.com/url?sa=t&rct=j&q=&esrc=s&source=web&cd=&ved=2ahUKEwjZ8KyJmsCPAxV7LLkGHcMYE9gQFnoECBgQAQ&url=https%3A%2F%2Fsol.sbc.org.br%2Findex.php%2Fbresci%2Farticle%2Fdownload%2F30591%2F30395%2F&usg=AOvVaw26Lp46JpIKoVVGP4wQQEU7&opi=89978449}{SBC (anais)}
\end{itemize}

\subsection{Guimarães et al. (2024) -- Brazilian J. Implantology \& Health Sciences}
\begin{itemize}[label={}, leftmargin=0cm, itemindent=2cm, align=left]
  \item[\textbf{Objetivo:}] Revisar diagnóstico e tratamento da asma.
  \item[\textbf{Metodologia:}] Revisão integrativa recente de diretrizes e práticas.
  \item[\textbf{Resultados:}] Síntese de estratégias diagnósticas e terapias farmacológicas/não farmacológicas.
  \item[\textbf{Principal contribuição:}] Atualização prática para cenário brasileiro.
  \item[\textbf{Link:}] \href{https://bjihs.emnuvens.com.br/bjihs/article/view/3228}{Artigo online}
\end{itemize}

\subsection{Hisinger-Molkånen et al. (2022) -- ERJ Open Research}
\begin{itemize}[label={}, leftmargin=0cm, itemindent=2cm, align=left]
  \item[\textbf{Objetivo:}] Relacionar idade ao diagnóstico com características da asma de início adulto.
  \item[\textbf{Metodologia:}] Estudos populacionais nórdicos, análise de fenótipo e gravidade.
  \item[\textbf{Resultados:}] Diagnóstico em idade mais avançada associa-se a sintomas mais graves e pior controle.
  \item[\textbf{Principal contribuição:}] Salienta relevância do diagnóstico precoce para melhor prognóstico.
  \item[\textbf{Link:}] \href{https://pubmed.ncbi.nlm.nih.gov/36185544}{Pubmed}
\end{itemize}

\subsection{IBGE (2020) -- PNS 2019 (Relatório)}
\begin{itemize}[label={}, leftmargin=0cm, itemindent=2cm, align=left]
  \item[\textbf{Conteúdo:}] Metodologia e resultados sobre saúde, estilos de vida e doenças crônicas no Brasil.
  \item[\textbf{Utilidade:}] Fonte oficial para estimar prevalência autorreferida de asma e determinantes sociais.
  \item[\textbf{Link:}] \href{https://biblioteca.ibge.gov.br/visualizacao/livros/liv101764.pdf}{Relatório PNS 2019 (PDF)}
\end{itemize}

\subsection{Kaplan et al. (2021) -- JACI: In Practice}
\begin{itemize}[label={}, leftmargin=0cm, itemindent=2cm, align=left]
  \item[\textbf{Objetivo:}] Revisar aplicações de IA/ML em pneumologia com foco em asma e DPOC.
  \item[\textbf{Metodologia:}] Revisão narrativa de avanços e lacunas.
  \item[\textbf{Resultados:}] Potenciais em imagem, função pulmonar e suporte diagnóstico.
  \item[\textbf{Principal contribuição:}] Agenda para incorporação responsável de IA no cuidado respiratório.
  \item[\textbf{Link:}] \href{https://pubmed.ncbi.nlm.nih.gov/33618053/}{PubMed} \,|\, \href{https://doi.org/10.1016/j.jaip.2021.02.014}{DOI}
\end{itemize}

\subsection{Lorensia, Suryadinata \& Saputra (2019) -- J. Pharm. Sci. \& Research}
\begin{itemize}[label={}, leftmargin=0cm, itemindent=2cm, align=left]
  \item[\textbf{Objetivo:}] Avaliar relação entre atividade física, vitamina D e asma.
  \item[\textbf{Metodologia:}] Estudo comparativo entre asmáticos e não asmáticos.
  \item[\textbf{Resultados:}] Níveis mais baixos de vitamina D nos asmáticos; papel potencial da atividade física.
  \item[\textbf{Principal contribuição:}] Sugere interação nutrição–atividade física no fenótipo asmático.
  \item[\textbf{Link:}] \href{https://www.researchgate.net/profile/Amelia-Lorensia/publication/333745187_Physical_Activity_and_Vitamin_D_Level_in_Asthma_and_Non-Asthma/links/5d01aec64585157d15a6a73e/Physical-Activity-and-Vitamin-D-Level-in-Asthma-and-Non-Asthma.pdf}{PDF (revista)}
\end{itemize}

\subsection{Lual \& Awoke (2021) -- Journal of Asthma and Allergy}
\begin{itemize}[label={}, leftmargin=0cm, itemindent=2cm, align=left]
  \item[\textbf{Objetivo:}] Identificar fatores do controle subótimo da asma em adultos.
  \item[\textbf{Metodologia:}] Estudo transversal com pacientes; regressão multivariada.
  \item[\textbf{Resultados:}] Baixa adesão, comorbidades e fatores socioeconômicos associados a pior controle.
  \item[\textbf{Principal contribuição:}] Evidencia determinantes sociais e comportamentais no manejo.
  \item[\textbf{Link:}] \href{https://pubmed.ncbi.nlm.nih.gov/34497674}{Texto completo}
\end{itemize}

\subsection{Meltzer et al. (2019) -- Journal of Asthma}
\begin{itemize}[label={}, leftmargin=0cm, itemindent=2cm, align=left]
  \item[\textbf{Objetivo:}] Investigar papel do sono em desfechos de asma em adolescentes.
  \item[\textbf{Metodologia:}] Estudo observacional com medidas de sono e eventos de asma.
  \item[\textbf{Resultados:}] Menos horas de sono associadas a pior controle/maior exacerbação.
  \item[\textbf{Principal contribuição:}] Consolida o sono como alvo de intervenção em jovens com asma.
  \item[\textbf{Link:}] \href{https://pubmed.ncbi.nlm.nih.gov/31751908}{PubMed} 
\end{itemize}

\subsection{Menezes et al. (2015) -- J. Bras. Pneumol.}
\begin{itemize}[label={}, leftmargin=0cm, itemindent=2cm, align=left]
  \item[\textbf{Objetivo:}] Estimar prevalência de asma em adultos (PNS 2013).
  \item[\textbf{Metodologia:}] Inquérito populacional representativo.
  \item[\textbf{Resultados:}] Prevalência variou por sexo, idade e região.
  \item[\textbf{Principal contribuição:}] Marco inicial para adultos no Brasil, servindo de linha de base.
  \item[\textbf{Link:}] \href{https://doi.org/10.1590/1980-5497201500060018}{SciELO}
\end{itemize}

\subsection{Ramos, Martins \& Castro (2021) -- \emph{Physis: Revista de Saúde Coletiva}}
\begin{itemize}[label={}, leftmargin=0cm, itemindent=2cm, align=left]
  \item[\textbf{Objetivo:}] Revisar prevalência de asma por regiões do Brasil.
  \item[\textbf{Metodologia:}] Revisão sistemática de estudos regionais.
  \item[\textbf{Resultados:}] Variabilidade importante entre as cinco regiões.
  \item[\textbf{Principal contribuição:}] Panorama comparativo regional para informar alocação de recursos.
  \item[\textbf{Link:}] \href{https://ojs.brazilianjournals.com.br/ojs/index.php/BJHR/article/view/30260/pdf}{link}
\end{itemize}

\subsection{Razavi-Termeh, Sadeghi-Niaraki \& Choi (2021) -- ISPRS IJGI}
\begin{itemize}[label={}, leftmargin=0cm, itemindent=2cm, align=left]
  \item[\textbf{Objetivo:}] Mapear áreas suscetíveis à asma com técnicas de mineração de dados.
  \item[\textbf{Metodologia:}] Modelagem preditiva com dados ambientais/saúde; algoritmos \emph{ensemble}.
  \item[\textbf{Resultados:}] Mapas identificaram áreas de maior risco em cenário urbano.
  \item[\textbf{Principal contribuição:}] Integra geoinformação e IA à vigilância epidemiológica.
  \item[\textbf{Link:}] \href{https://www.mdpi.com/2072-4292/13/16/3222}{Texto}
\end{itemize}

\subsection{Rodrigues et al. (2021) -- Revista Brasileira de Medicina (narrativa)}
\begin{itemize}[label={}, leftmargin=0cm, itemindent=2cm, align=left]
  \item[\textbf{Objetivo:}] Descrever abordagem geral da asma (diagnóstico e terapias).
  \item[\textbf{Metodologia:}] Revisão narrativa.
  \item[\textbf{Resultados:}] Síntese de diretrizes e desafios práticos no cuidado.
  \item[\textbf{Principal contribuição:}] Referência didática para clínicos.
  \item[\textbf{Link:}] \href{https://acervomais.com.br/index.php/medico/article/download/9129/5572/}{PDF (acervo)}
\end{itemize}

\subsection{Xavier et al. (2022) -- Revista Paulista de Pediatria}
\begin{itemize}[label={}, leftmargin=0cm, itemindent=2cm, align=left]
  \item[\textbf{Objetivo:}] Avaliar sazonalidade climática e internações por doenças respiratórias em crianças.
  \item[\textbf{Metodologia:}] Modelo preditivo com dados de hospitalizações.
  \item[\textbf{Resultados:}] Sazonalidade associada a aumento de internações em períodos específicos.
  \item[\textbf{Principal contribuição:}] Evidencia papel de fatores climáticos na carga pediátrica (inclui asma).
  \item[\textbf{Link:}] \href{https://pubmed.ncbi.nlm.nih.gov/37097057}{PubMed}
\end{itemize}

% ----------------------------------------------------------
% REFERÊNCIAS (com links)
% ----------------------------------------------------------
\section{Referências}
\begin{itemize}[leftmargin=0.5cm]
    \item Estanislau, N.R.A. \emph{et al.} (2021). Association between asthma and sleep hours in Brazilian adolescents: ERICA. \textit{Jornal de Pediatria}, 97(4):396--401. \href{https://pubmed.ncbi.nlm.nih.gov/32956628/}{Link} \,|\, \href{https://doi.org/10.1016/j.jped.2020.07.007}{DOI}.
    \item Finkelstein, J.; Jeong, I.C. (2017). Machine learning approaches to personalize early prediction of asthma exacerbations. \textit{Ann NY Acad Sci} 1387(1):153--165. \href{https://pubmed.ncbi.nlm.nih.gov/27627195}{Link}
    \item Gonçalves, L.; França, D.; Zarate, L. (2024). Relevância do entendimento do domínio... \textit{XVIII Brazilian e-Science Workshop}. \href{https://www.google.com/url?sa=t&rct=j&q=&esrc=s&source=web&cd=&ved=2ahUKEwjZ8KyJmsCPAxV7LLkGHcMYE9gQFnoECBgQAQ&url=https%3A%2F%2Fsol.sbc.org.br%2Findex.php%2Fbresci%2Farticle%2Fdownload%2F30591%2F30395%2F&usg=AOvVaw26Lp46JpIKoVVGP4wQQEU7&opi=89978449}{Link}.
    \item Guimarães, A.C.C.M. \emph{et al.} (2024). Diagnóstico e tratamento da asma: uma revisão. \textit{Braz. J. Implantology \& Health Sciences}, 6(8):5230--5240. \href{https://bjihs.emnuvens.com.br/bjihs/article/view/3228}{Link}.
    \item Hisinger-Molkånen, H. \emph{et al.} (2022). Age at diagnosis and disease characteristics of adult-onset asthma. \textit{ERJ Open Research}, 8(1). \href{https://pubmed.ncbi.nlm.nih.gov/36185544}{Link} 
    \item IBGE (2020). PNS 2019: Percepção do estado de saúde... Rio de Janeiro: IBGE. \href{https://biblioteca.ibge.gov.br/visualizacao/livros/liv101764.pdf}{PDF}.
    \item Kaplan, A. \emph{et al.} (2021). AI/ML in respiratory medicine and potential role in asthma/COPD diagnosis. \textit{JACI: In Practice}, 9(6):2255--2261. \href{https://pubmed.ncbi.nlm.nih.gov/33618053/}{Link} \,|\, \href{https://doi.org/10.1016/j.jaip.2021.02.014}{DOI}.
    \item Lorensia, A.; Suryadinata, R.V.; Saputra, A. (2019). Physical activity, vitamin D and asthma. \textit{J Pharm Sci Res}, 11(2):507--510. \href{https://www.researchgate.net/profile/Amelia-Lorensia/publication/333745187_Physical_Activity_and_Vitamin_D_Level_in_Asthma_and_Non-Asthma/links/5d01aec64585157d15a6a73e/Physical-Activity-and-Vitamin-D-Level-in-Asthma-and-Non-Asthma.pdf}{PDF}.
    \item Lual, N.; Awoke, T. (2021). Factors associated with suboptimal asthma control among adults. \textit{Journal of Asthma and Allergy}, 14:1033--1042. \href{https://pubmed.ncbi.nlm.nih.gov/34497674}{Link}.
    \item Meltzer, L.J. \emph{et al.} (2019). The role of sleep in asthma outcomes in adolescents. \textit{Journal of Asthma}, 56(9):957--965. \href{https://pubmed.ncbi.nlm.nih.gov/31751908}{Link}
    \item Menezes, A.M.B. \emph{et al.} (2015). Prevalence of asthma in adults in Brazil: PNS 2013. \textit{J Bras Pneumol}, 41(5):398--404. \href{https://doi.org/10.1590/1980-5497201500060018}{SciELO}.
    \item Ramos, D.; Martins, D.; Castro, M. (2021). Prevalência de asma nas regiões do Brasil: revisão sistemática. \textit{Physis}, 31(2):e310220. \href{https://ojs.brazilianjournals.com.br/ojs/index.php/BJHR/article/view/30260/pdf}{link}.
    \item Razavi-Termeh, S.V.; Sadeghi-Niaraki, A.; Choi, S.-M. (2021). Asthma susceptible area map with data mining. \textit{ISPRS IJGI}, 10(3):178. \href{https://www.mdpi.com/2072-4292/13/16/3222}{Link} 
    \item Rodrigues, L. \emph{et al.} (2021). Asma: uma revisão narrativa. \textit{Revista Brasileira de Medicina}, 78(5):45--53. \href{https://acervomais.com.br/index.php/medico/article/download/9129/5572/}{PDF}.
    \item Xavier, C.C. \emph{et al.} (2022). Seasonal variation and pediatric respiratory admissions. \textit{Rev Paul Pediatr}, 40:e2020186. \href{https://pubmed.ncbi.nlm.nih.gov/37097057}{PubMed}
\end{itemize}

\end{document}
