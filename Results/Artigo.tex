\documentclass[12pt,a4paper]{article}

% -----------------------------
% Pacotes de configuração
% -----------------------------
\usepackage[utf8]{inputenc}   
\usepackage[T1]{fontenc}      
\usepackage[english,brazilian]{babel} 
\usepackage[left=3cm, right=2cm, top=3cm, bottom=2cm]{geometry} 
\usepackage{setspace}         
\usepackage{indentfirst}      
\usepackage{microtype}        
\usepackage{graphicx}         % Para incluir gráficos e figuras
\usepackage{float}            % Para posicionamento de figuras
\usepackage{hyperref}         % Links clicáveis no PDF

\begin{document}

% ----------------------------------------------------------
% FOLHA DE ROSTO
% ----------------------------------------------------------
\begin{titlepage}
    \begin{center}
        \vspace*{4cm}
        \textbf{\MakeUppercase{Acidente Vascular Cerebral no Brasil: o que revelam os dados da PNS 2019 sobre alimentação, estilo de vida e acesso à saúde}}
        
        \vspace{3cm}
    \end{center}
    
    \begin{flushright}
        Isaque Gomes Azevedo \\
        \texttt{isaquegomes623@gmail.com}
    \end{flushright}
    
    \vfill
    
    \begin{center}
        Belo Horizonte \\
        2025
    \end{center}
\end{titlepage}

\pagestyle{plain} % Numeração a partir daqui

% ----------------------------------------------------------
% SUMÁRIO
% ----------------------------------------------------------
\newpage
\tableofcontents
\newpage

% ----------------------------------------------------------
% RESUMO
% ----------------------------------------------------------
\section*{Resumo}
\addcontentsline{toc}{section}{Resumo}
\begin{singlespace}
\noindent
\textbf{Contexto:} O Acidente Vascular Cerebral (AVC) é uma das principais causas de morbimortalidade no Brasil. Fatores relacionados à alimentação, estilo de vida, determinantes sociais e acesso a serviços de saúde desempenham papel fundamental na ocorrência e no manejo dessa condição. 

\textbf{Objetivo:} Descrever o perfil de adultos brasileiros com diagnóstico médico autorreferido de AVC na Pesquisa Nacional de Saúde (PNS) 2019 e analisar associações com alimentação, estilo de vida, determinantes sociais e acesso a serviços de saúde. 

\textbf{Métodos:} Estudo transversal com dados da PNS 2019, incluindo todos os indivíduos. O desfecho foi o autorrelato de diagnóstico médico de AVC. Variáveis independentes incluíram marcadores de alimentação (consumo de frutas, legumes, ultraprocessados, carne vermelha), estilo de vida (tabagismo, consumo de álcool, prática de atividade física), determinantes sociais (sexo, escolaridade, renda, região, raça/cor) e acesso a serviços de saúde (consultas médicas, plano de saúde, uso do SUS). Foram conduzidas análises descritivas e modelos de regressão logística (brutos e ajustados) considerando o desenho amostral complexo e pesos da pesquisa. 

\textbf{Discussão:} Os achados serão interpretados à luz das desigualdades sociais em saúde e da literatura sobre fatores de risco cerebrovascular. Por ser estudo transversal, os resultados descrevem associações e perfis, sem permitir inferências de causalidade ou predição de risco futuro. 

\textbf{Implicações:} O estudo pode subsidiar políticas públicas de prevenção e promoção da saúde cerebrovascular, especialmente em grupos vulneráveis.
\vspace{\baselineskip}

\noindent
\textbf{Palavras-chave:} Acidente Vascular Cerebral; alimentação; estilo de vida; determinantes sociais da saúde; acesso a serviços de saúde.
\end{singlespace}

\newpage

% ----------------------------------------------------------
% INTRODUÇÃO
% ----------------------------------------------------------
\section{Introdução}
O Acidente Vascular Cerebral (AVC) é uma condição de alta morbimortalidade no Brasil e no mundo. Sua ocorrência está associada a fatores de risco modificáveis, como hábitos alimentares, sedentarismo, tabagismo, consumo de álcool, além de determinantes sociais e desigualdades no acesso a serviços de saúde. A compreensão do perfil de indivíduos com AVC é fundamental para orientar políticas públicas de prevenção e promoção da saúde.

% ----------------------------------------------------------
% METODOLOGIA
% ----------------------------------------------------------
\section{Metodologia}
\subsection{Base de dados}
Os dados foram obtidos da Pesquisa Nacional de Saúde (PNS) 2019, que inclui informações sobre saúde, estilo de vida, alimentação, acesso a serviços de saúde e condições socioeconômicas da população brasileira.

\subsection{População do estudo}
Incluiu todos os indivíduos com diagnóstico médico autorreferido de AVC, sem restrição de faixa etária.

\subsection{Variáveis}
\begin{itemize}
    \item \textbf{Desfecho:} Diagnóstico médico de AVC (autorreferido).
    \item \textbf{Variáveis independentes:} Alimentação (frutas, legumes, carne vermelha, ultraprocessados), estilo de vida (tabagismo, consumo de álcool, atividade física), determinantes sociais (sexo, escolaridade, renda, região, raça/cor) e acesso a serviços de saúde (consultas médicas, plano de saúde, uso do SUS).
\end{itemize}

\subsection{Análise estatística}
Foram realizadas análises descritivas por UF e região, com frequências absoluta e relativa, além de gráficos. Modelos de regressão logística foram estimados para avaliar associações brutas e ajustadas, considerando o desenho amostral complexo e pesos da pesquisa.

% ----------------------------------------------------------
% RESULTADOS
% ----------------------------------------------------------
\section{Resultados}
\subsection{Diagnóstico por Unidade da Federação (UF)}
Aqui você pode incluir os gráficos por UF. Por exemplo:
\begin{figure}[H]
    \centering
    \includegraphics[width=0.8\textwidth]{grafico_uf.png} % Substitua pelo arquivo gerado em Python
    \caption{Distribuição de casos de AVC por UF (absoluto e relativo).}
    \label{fig:uf}
\end{figure}

\subsection{Diagnóstico por Região}
\begin{figure}[H]
    \centering
    \includegraphics[width=0.6\textwidth]{grafico_regiao.png} % Substitua pelo arquivo gerado em Python
    \caption{Distribuição de casos de AVC por região (absoluto e relativo).}
    \label{fig:regiao}
\end{figure}

\subsection{Outras análises descritivas}
Você pode adicionar tabelas de frequência de sexo, escolaridade, renda, hábitos alimentares, etc.

% ----------------------------------------------------------
% DISCUSSÃO
% ----------------------------------------------------------
\section{Discussão}
Os resultados evidenciam as desigualdades regionais e sociais no diagnóstico de AVC. A análise reforça a importância de políticas públicas direcionadas à prevenção de fatores de risco e ao acesso igualitário a serviços de saúde. A interpretação deve considerar a natureza transversal do estudo.

% ----------------------------------------------------------
% CONCLUSÃO
% ----------------------------------------------------------
\section{Conclusão}
Este estudo descreve o perfil epidemiológico de AVC no Brasil usando dados da PNS 2019. Os achados destacam a necessidade de estratégias de saúde pública para prevenção e manejo da doença, especialmente entre grupos vulneráveis.

% ----------------------------------------------------------
% REFERÊNCIAS
% ----------------------------------------------------------
\section{Referências}
\begin{itemize}
    \item Ministério da Saúde. Pesquisa Nacional de Saúde 2019. Disponível em: \url{https://www.ibge.gov.br/estatisticas/sociais/saude/21892-pns.html}
    \item World Health Organization. Stroke. Disponível em: \url{https://www.who.int/news-room/fact-sheets/detail/stroke}
\end{itemize}

\end{document}
