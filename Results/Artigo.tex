\documentclass[12pt,a4paper]{article}

% -----------------------------
% Pacotes
% -----------------------------
\usepackage[utf8]{inputenc}
\usepackage[T1]{fontenc}
\usepackage[english,brazilian]{babel}
\usepackage[left=3cm, right=2cm, top=3cm, bottom=2cm]{geometry}
\usepackage{setspace}
\usepackage{indentfirst}
\usepackage{microtype}
\usepackage{graphicx}
\usepackage{float}
\usepackage{hyperref}
\usepackage{enumitem}
\usepackage{longtable}
\usepackage{booktabs}
\usepackage{siunitx}
\usepackage[table]{xcolor}

\sisetup{
  round-mode=places,
  round-precision=3,
  detect-weight=true,
  detect-inline-weight=math,
  group-separator = {.},
  output-decimal-marker = {,},
  table-number-alignment = center
}
\hypersetup{
  colorlinks=true,
  urlcolor=blue,
  citecolor=black,
  linkcolor=black,
  pdfauthor={Isaque Gomes Azevedo},
  pdftitle={Asma no Brasil: o que revelam os dados da PNS 2019}
}

\begin{document}

% -----------------------------
% Folha de rosto
% -----------------------------
\begin{titlepage}
    \begin{center}
        \vspace*{4cm}
        \textbf{\MakeUppercase{Asma no Brasil: o que revelam os dados da PNS 2019 sobre alimentação, estilo de vida e acesso à saúde}}
        \vspace{3cm}
    \end{center}

    \begin{flushright}
        \textbf{Autor:} Isaque Gomes Azevedo \\
        \texttt{isaquegomes623@gmail.com} \\[1.5em]
        \textbf{Orientador:} Prof. Luis Enrique Zarate Galvez
    \end{flushright}

    \vfill
    \begin{center}
        Belo Horizonte \\
        2025
    \end{center}
\end{titlepage}

\pagestyle{plain}


\pagestyle{plain}

% -----------------------------
% Sumário
% -----------------------------
\newpage
\tableofcontents
\newpage

% -----------------------------
% Resumo
% -----------------------------
% -----------------------------
% Resumo (versão revisada)
% -----------------------------
\section*{Resumo}
\addcontentsline{toc}{section}{Resumo}
\begin{singlespace}
\noindent
\textbf{Contexto:}
A asma é uma doença crônica prevalente e desigual no Brasil. Fatores alimentares e comportamentais, determinantes sociais e acesso aos serviços influenciam ocorrência e controle.

\vspace{0.5em}
\textbf{Objetivo:}
Estimar a prevalência autorreferida de diagnóstico médico de asma entre adultos na PNS 2019 e investigar associações com marcadores de alimentação, estilo de vida, determinantes sociais e acesso/uso de serviços, identificando gradientes sociais e grupos vulneráveis. Análises secundárias exploraram a faixa etária 0–30 anos.

\vspace{0.5em}
\textbf{Métodos:}
Inquérito transversal com microdados públicos da PNS 2019 (IBGE), considerando o desenho amostral complexo (pesos, estratos e UPA). O desfecho foi o diagnóstico médico de asma autorreferido. Modelos logísticos hierarquizados (blocos: sociodemográfico; estilo de vida; alimentação; acesso/uso), com efeitos marginais ajustados. Tratamento de dependências do questionário: codificação explícita de “não se aplica” (NSA=0) e imputação parcimoniosa (média no grupo aplicável para variáveis numéricas; moda para categóricas). Sensibilidades: exclusão de DPOC/bronquite/enfisema; recortes por sexo/idade; reclassificações alimentares.

\vspace{0.5em}
\textbf{Resultados:}
Foram analisados [N não ponderado] adultos (representando [X] milhões). A prevalência de asma foi de [P\%] (IC95\% [L--U]). Observaram-se gradientes por escolaridade e renda (ex.: baixa vs. alta escolaridade: OR ajustada [OR\_esc], IC95\% [L--U]) e associação com tabagismo atual (OR [OR\_tab], IC95\% [L--U]) e menor atividade física (OR [OR\_af], IC95\% [L--U]). A subanálise 0–30 anos confirmou maior concentração de diagnósticos precoces. Resultados foram estáveis a sensibilidades e à imputação.

\vspace{0.5em}
\textbf{Discussão:}
Achados indicam que condições de vida, comportamentos e acesso/uso de serviços se inter-relacionam na determinação e no manejo da asma, em linha com literatura de determinantes sociais.

\vspace{0.5em}
\textbf{Implicações:}
Reforço da APS/SUS com protocolos ativos de asma, territorialização do risco, ambientes livres de fumo e promoção de alimentação in natura/atividade física, com monitoramento integrado.

\vspace{\baselineskip}
\noindent
\textbf{Palavras-chave:} Asma; determinantes sociais; alimentação; estilo de vida; acesso aos serviços de saúde; PNS 2019.
\end{singlespace}
\newpage


\newpage

% -----------------------------
% Introdução
% -----------------------------
% -----------------------------
% Introdução (versão revisada)
% -----------------------------
\section{Introdução}
A asma é uma doença inflamatória crônica das vias aéreas, caracterizada por exacerbações recorrentes de dispneia, sibilos e tosse, associadas à hiperresponsividade brônquica. Além da suscetibilidade genética e exposições ambientais, fatores modificáveis — como tabagismo, dieta de baixa qualidade e inatividade física — e determinantes sociais (escolaridade, renda, ocupação, raça/cor) influenciam tanto o risco quanto o controle da doença. Barreiras no acesso/uso de serviços de saúde também podem agravar desfechos.

No Brasil, inquéritos nacionais como a Pesquisa Nacional de Saúde (PNS) permitem estimar prevalências e monitorar desigualdades entre regiões e grupos populacionais, oferecendo subsídios para o planejamento em saúde. Embora existam estudos sobre asma e alguns de seus determinantes, ainda são escassas análises que integrem, em um único arcabouço, alimentação, estilo de vida, determinantes sociais e acesso/uso de serviços na PNS 2019, com modelagem hierarquizada e avaliação explícita de dependências do questionário.

\textbf{Objetivo:} estimar a prevalência autorreferida de asma entre adultos na PNS 2019 e investigar associações com padrões alimentares, estilo de vida, determinantes sociais e acesso/uso de serviços, buscando gradientes sociais e grupos vulneráveis. Análises secundárias exploram a concentração de diagnósticos em idades precoces (0–30 anos).


% -----------------------------
% Metodologia
% -----------------------------
% -----------------------------
% Metodologia (versão consolidada)
% -----------------------------
\section{Metodologia}

\subsection{Desenho e fonte de dados}
Estudo transversal com microdados públicos da PNS 2019 (IBGE/Ministério da Saúde), amostra probabilística e representatividade nacional. As informações metodológicas detalhadas constam no relatório oficial da PNS 2019.

\subsection{População, elegibilidade e subanálises}
Incluímos todos os \textbf{adultos (≥18 anos)} com dados válidos para o desfecho e covariáveis essenciais. O desfecho foi o \textbf{diagnóstico médico de asma autorreferido}.
Como \textbf{subanálise exploratória}, investigamos distribuição etária e concentração de diagnósticos no intervalo 0–30 anos, motivada pela maior incidência em idades precoces.

\subsection{Variáveis e definições}
\textbf{Desfecho:} asma (autorreferida).
\par\noindent
\textbf{Exposições principais (por blocos):}
\begin{enumerate}[noitemsep, topsep=0pt]
    \item \textit{Sociodemográficas e contexto:} sexo, idade, raça/cor, escolaridade, renda per capita, ocupação, macrorregião, zona urbano/rural.
    \item \textit{Estilo de vida:} tabagismo (atual/ex), consumo de álcool, atividade física no lazer.
    \item \textit{Alimentação:} consumo de frutas/legumes, carnes, bebidas açucaradas e ultraprocessados (itens da PNS com marcadores semanais e no dia anterior).
    \item \textit{Acesso/uso de serviços:} plano de saúde, consulta nos últimos 12 meses, uso do SUS, procura por atendimento.
\end{enumerate}

\subsection{Plano amostral e ponderação}
Todas as estimativas consideraram pesos amostrais, estratos e \textit{UPA/PSU} do desenho complexo. As análises foram conduzidas em [\textit{R/Stata/Python}] com pacotes específicos para dados complexos (ex.: \texttt{survey}/\texttt{svy}).

\subsection{Aspectos éticos e reprodutibilidade}
A PNS possui aprovação ética nacional; microdados são públicos e anonimizados. Código analítico e dicionário de tratamento de dependências/imputação disponíveis em [\textit{repositório a ser informado}] ou sob solicitação.

% -----------------------------
% Resultados
% -----------------------------
\section{Resultados}

\subsection{Panorama nacional e por UF}
\begin{figure}[H]
    \centering
    \includegraphics[width=0.85\textwidth]{diagnostico_por_uf_regiao-1.pdf}
    \includegraphics[width=0.85\textwidth]{diagnostico_por_uf_regiao-2.pdf}
    \caption{Prevalência autorreferida de asma por UF (absoluto e relativo, IC95\%).}
    \label{fig:uf}
\end{figure}

\subsection{Diferenças regionais}
\begin{figure}[H]
    \centering
    \includegraphics[width=0.85\textwidth]{diagnostico_por_uf_regiao-3.pdf}
    \includegraphics[width=0.85\textwidth]{diagnostico_por_uf_regiao-4.pdf}
    \caption{Distribuição por macrorregião (absoluto e relativo).}
    \label{fig:regioes}
\end{figure}

\subsection{População saudável vs. asmática}
\begin{figure}[H]
    \centering
    \includegraphics[width=0.8\textwidth]{saudaveis_vs_asma_0_30.png}
    \caption{Comparação entre indivíduos saudáveis e doentes (asma) na faixa etária de 0 a 30 anos – PNS 2019.}
    \label{fig:saudaveis_asma}
\end{figure}

A Figura~\ref{fig:saudaveis_asma} mostra que a população saudável na faixa etária de 0 a 30 anos é majoritária em comparação com os indivíduos que referiram diagnóstico de asma. Foram identificados aproximadamente \textbf{19.465 saudáveis} contra apenas \textbf{cerca de 1.188 doentes}, evidenciando a baixa prevalência relativa da doença nesse grupo etário.

\subsection{Distribuição etária dos asmáticos}

\begin{figure}[H]
    \centering
    \includegraphics[width=0.95\textwidth]{idade_asma.png}
    \caption{Distribuição da idade das pessoas com diagnóstico de asma — PNS 2019.}
    \label{fig:idade_asma}
\end{figure}

A Figura~\ref{fig:idade_asma} mostra a distribuição etária dos indivíduos com diagnóstico de asma. 
Observa-se um pico de frequência nas idades mais jovens, especialmente até os \textbf{10 anos}, seguido por uma queda gradual a partir da adolescência. 
Esse padrão evidencia que a asma apresenta maior concentração nos primeiros anos de vida, com diminuição progressiva da ocorrência em faixas etárias mais elevadas.

\begin{figure}[H]
    \centering
    \includegraphics[width=0.95\textwidth]{frequencia_acumulada_simples_asma.png}
    \caption{Frequência acumulada simples de pessoas com asma por idade — PNS 2019.}
    \label{fig:frequencia_acumulada_asma}
\end{figure}

A Figura~\ref{fig:frequencia_acumulada_asma} apresenta a curva de frequência acumulada simples. 
Nota-se um crescimento acentuado até aproximadamente os \textbf{30 anos}, quando a curva tende a estabilizar-se. 
Isso indica que a maior parte dos diagnósticos ocorre até essa idade, representando a fase de maior incidência da doença. 
Após os 30 anos, a ocorrência de novos casos torna-se rara, predominando indivíduos que já convivem com a condição.

Com base nessas observações, a análise será concentrada na faixa etária de \textbf{0 a 30 anos}. 
Esse intervalo engloba o período de maior incidência e surgimento da asma, contemplando tanto a infância quanto a transição para a vida adulta. 
A curva acumulada evidencia que cerca de 90\% dos casos estão concentrados nesse intervalo, reforçando que a inclusão de idades superiores tenderia apenas a adicionar casos residuais, com pouca relevância para a compreensão do padrão de incidência.

\subsection{Modelo conceitual}
\begin{figure}[H]
    \centering
    \includegraphics[width=0.95\textwidth]{Group 2.png}
    \caption{Mapa conceitual das dimensões, aspectos e atributos relacionados à asma na PNS 2019.}
    \label{fig:mapa}
\end{figure}

\subsection{Tabela de Dimensões e Atributos (base revisada PNS 2019)}
\renewcommand{\arraystretch}{1.35}
\setlength{\tabcolsep}{6pt}
\begin{longtable}{|p{4cm}|p{11cm}|}
\hline
\textbf{Dimensão} & \textbf{Atributos (código – descrição resumida)} \\
\hline
\endfirsthead
\hline
\textbf{Dimensão} & \textbf{Atributos (código – descrição resumida)} \\
\hline
\endhead

% ---------------------- DETERMINANTES SOCIAIS ----------------------
\textbf{Determinantes sociais e domicílio} &
A001 – Tipo de domicílio;  
A002010 – Material predominante nas paredes externas;  
A003010 – Material predominante no telhado;  
A004010 – Material predominante no piso;  
A005010 – Forma de abastecimento de água;  
A005012 – Ligado à rede pública de água;  
A009010 – Tipo de água utilizada para beber;  
A02201 – Presença de animal de estimação;  
C001 – Número de pessoas no domicílio;  
C006 – Sexo;  
C008 – Idade do morador;  
C009 – Cor ou raça;  
V0001 – Unidade da Federação;  
V0022 – Total de moradores;  
V0026 – Situação censitária (urbano/rural). \\
\hline

% ---------------------- FATORES AMBIENTAIS ----------------------
\textbf{Fatores ambientais e contexto} &
A02305 – Quantos gatos;  
A02306 – Quantos cachorros;  
A02307 – Quantas aves;  
P068 – Frequência de fumo dentro do domicílio. \\
\hline

% ---------------------- ALIMENTAÇÃO ----------------------
\textbf{Alimentação} &
P006 – Dias por semana em que come feijão;  
P00601–P00623 – Consumo de alimentos no dia anterior (grupos alimentares e ultraprocessados);  
P00901 – Dias comendo verduras/legumes;  
P01001 – Forma de consumo de verduras;  
P01101 – Dias comendo carne vermelha;  
P013 – Dias comendo frango;  
P015 – Dias comendo peixe;  
P01601 – Dias tomando suco natural;  
P018 – Dias comendo frutas;  
P019 – Vezes que come frutas por dia;  
P02001 – Dias tomando suco industrializado;  
P02002 – Dias tomando refrigerante;  
P02101 – Tipo de suco;  
P02102 – Tipo de refrigerante;  
P023 – Dias tomando leite;  
P02401 – Tipo de leite;  
P02501 – Dias comendo doces;  
P02602 – Dias que substitui almoço por lanches. \\
\hline

% ---------------------- ESTILO DE VIDA ----------------------
\textbf{Estilo de vida e comportamento} &
P050 – Fuma atualmente;  
P051 – Já fumou diariamente;  
P052 – Já fumou algum produto do tabaco;  
P053 – Idade de início do tabagismo;  
P05401–P05416 – Tipo e quantidade de tabaco consumido;  
P058 – Quantos cigarros fumava por dia/semana;  
P05901–P05904 – Tempo desde que parou de fumar;  
P067 – Uso de tabaco sem fumaça;  
P06701 – Uso de cigarros eletrônicos;  
P034 – Prática de exercício físico. \\
\hline

% ---------------------- CONDIÇÕES CLÍNICAS ----------------------
\textbf{Condições clínicas} &
Q074 – Diagnóstico médico de asma;  
Q075 – Idade ao diagnóstico de asma;  
Q076 – Crise de asma nos últimos 12 meses;  
Q07601 – Uso de medicamento para asma;  
Q07704 – Uso de medicamentos orais (2 semanas);  
Q07708 – Uso de aerossol (bombinha);  
Q11605 – Diagnóstico de enfisema pulmonar;  
Q11606 – Diagnóstico de bronquite crônica;  
Q11607 – Outro diagnóstico pulmonar. \\
\hline

% ---------------------- ANTROPOMETRIA ----------------------
\textbf{Antropometria} &
P00103 – Peso informado (kg);  
P00104 – Peso final (kg);  
P00403 – Altura informada (cm);  
P00404 – Altura final (cm). \\
\hline

% ---------------------- INFÂNCIA / AMAMENTAÇÃO ----------------------
\textbf{Infância e amamentação} &
L01701–L01716 – Introdução alimentar (leite, frutas, vegetais, doces, refrigerantes, etc.);  
L019 – Recebeu sulfato ferroso. \\
\hline
\end{longtable}

\subsection{Tratamento e Padronização dos Dados}

O pré-processamento da base da PNS foi conduzido com foco na coerência lógica do questionário, na padronização das respostas e na eliminação de ambiguidades que pudessem comprometer análises estatísticas e modelos preditivos.

O ponto central foi o tratamento de \textbf{variáveis dependentes}, identificadas a partir do dicionário de dados no campo \textit{“Código da variável dependente”}. Em vez de assumir uma regra fixa (como considerar que a dependência ocorre apenas quando a variável base é igual a 1), o algoritmo realizou uma \textbf{detecção empírica dos valores habilitadores}. Para cada possível valor da variável condicionante, calculou-se a proporção de respostas válidas na variável dependente; os valores com maior consistência de preenchimento foram classificados como ativadores reais da pergunta. Esse procedimento eliminou suposições arbitrárias e garantiu aderência ao padrão concreto de resposta observado na pesquisa.

A partir desse mapeamento, duas regras de padronização foram aplicadas:

\begin{enumerate}
\item \textbf{Quando a condição de habilitação não era atendida}, a variável dependente recebeu o valor \textbf{0}, representando explicitamente “não se aplica”. Essa codificação evita que a ausência seja interpretada como dado faltante, preservando a lógica do instrumento de coleta.

\item \textbf{Quando a condição de habilitação era atendida, mas a resposta estava ausente}, realizou-se \textbf{imputação orientada ao tipo da variável}: média restrita ao grupo aplicável para variáveis numéricas e moda para variáveis categóricas. Em situações onde não havia dados suficientes no grupo habilitado, utilizou-se uma imputação de fallback (média ou moda global), sempre registrada de forma controlada.
\end{enumerate}

Para variáveis não dependentes, procedeu-se à padronização de respostas textuais, remoção de marcadores de ausência (\textit{NA}, \textit{nan}, campos vazios) e imputação segundo o mesmo critério (média para numéricos, moda para categóricos).

O resultado é uma base \textbf{completa, coerente e informativamente preservada}, na qual:

\begin{itemize}
\item “ausência de resposta” não é confundida com “não aplicável”;
\item as distribuições originais são mantidas sempre que possível;
\item as imputações não introduzem viés estrutural no padrão de respostas.
\end{itemize}

Esse tratamento fortalece a robustez das análises subsequentes, evita enviesamentos interpretativos e fornece uma matriz de dados adequada para métodos estatísticos e algoritmos de aprendizagem de máquina, respeitando o fluxo epidemiológico do questionário original.


\subsection{Resumo das Variaveis }
\input{tabelas_resumo.tex}

% -----------------------------
% Discussão
% -----------------------------
\section{Discussão}
Este estudo estimou prevalência de asma em [P\%] na população adulta brasileira (PNS 2019) e identificou gradientes sociais (escolaridade/renda/raça-cor), além de associações com comportamentos (tabagismo, atividade física) e marcadores alimentares. Em conjunto, os achados reforçam que condições de vida, rotinas e acesso/uso de serviços moldam a ocorrência e o manejo da asma.

Mecanisticamente, o tabagismo pode exacerbar inflamação das vias aéreas e hiperresponsividade, enquanto padrões alimentares marcados por ultraprocessados e bebidas açucaradas se associam a inflamação sistêmica de baixo grau. Atividade física insuficiente relaciona-se a piores perfis cardiorrespiratórios e maior sintomatologia. Determinantes sociais influenciam exposição a riscos ambientais, condições domiciliares e capacidade de acesso ao cuidado.

Nossos resultados são coerentes com estudos prévios nacionais e internacionais que identificam maior prevalência em grupos socialmente vulneráveis e associações com sono, atividade física e marcadores de dieta. Ao integrar alimentação, estilo de vida, determinantes sociais e acesso/uso de serviços em modelagem hierárquica com desenho amostral complexo, ampliamos a evidência específica para a PNS 2019 e explicitamos o papel da codificação “NSA=0” e de imputações parcimoniosas.

Entre as forças, destacam-se representatividade nacional, consideração do plano amostral, tratamento explícito de dependências do questionário e múltiplas sensibilidades. Entre as limitações, salientamos o caráter transversal (não permite inferência causal), autorrelato do desfecho e possíveis confundidores residuais (apesar do ajuste e diagnósticos). A imputação pode introduzir viés, mitigado por análises alternativas e registro da taxa de imputação.

Programaticamente, os achados sugerem priorização de grupos com menor escolaridade/renda e de territórios mais afetados, combinando ações de cessação do tabagismo, promoção de atividade física e alimentação in natura com qualificação do cuidado na APS/SUS e vigilância territorial.


% -----------------------------
% Conclusão (versão revisada)
% -----------------------------
\section{Conclusão}
A asma no Brasil apresenta gradientes sociais e associações consistentes com comportamentos e alimentação. Intervenções focadas em grupos de maior vulnerabilidade, reforço da APS, ambientes livres de fumo e promoção de hábitos saudáveis podem reduzir desigualdades e melhorar controle e qualidade de vida.

% -----------------------------
% Implicações para políticas públicas
% -----------------------------
\section{Implicações para políticas públicas}
\begin{itemize}[noitemsep]
  \item \textbf{APS com protocolos ativos de asma:} implementar plano de autocuidado e revisão medicamentosa semestral para ≥80\% dos asmáticos cadastrados (12 meses; responsável: coord. APS/UBS).
  \item \textbf{Territorialização do risco:} mapear trimestralmente prevalência autorreferida e internações por asma por UBS e microárea; priorizar visitas domiciliares onde a taxa for ≥P90 regional.
  \item \textbf{Ambientes livres de fumo:} garantir conformidade ≥90\% em escolas e repartições (inspeções semestrais; saúde+educação+trabalho).
  \item \textbf{Promoção de hábitos saudáveis:} grupos semanais de atividade física e educação alimentar (NASF/Academia da Saúde), com adesão ≥50\% dos inscritos e manutenção por ≥6 meses.
  \item \textbf{Monitoramento integrado:} painel unificado (e-SUS/SAE/SIH) com indicadores: prevalência, consultas, uso de medicação inalatória, internações; análise bimestral para realocar recursos.
\end{itemize}
% ----------------------------------------------------------
% REVISÃO DA LITERATURA — versão organizada por blocos
% ----------------------------------------------------------
\section{Revisão da Literatura}

\noindent
Esta revisão organiza a evidência em blocos temáticos que dialogam diretamente com o objetivo do estudo: (i) epidemiologia e desigualdades no Brasil; (ii) sono e asma; (iii) estilo de vida e alimentação; (iv) acesso/uso de serviços e manejo; (v) ciência de dados/IA aplicadas à asma; (vi) fenótipo e idade de início; (vii) risco espacial e geoinformação. Em cada bloco, sintetizamos contribuições, limitações e implicações para a modelagem proposta.

% ===================== BLOCO 1 =====================
\subsection{Epidemiologia e desigualdades no Brasil (PNS e estudos correlatos)}
\paragraph{IBGE (2020) -- PNS 2019 (Relatório).}
Fonte primária para prevalências e metodologia de inquérito, com escopo nacional e subsídios a recortes por região e determinantes sociais. Útil como base de calibração e \emph{benchmark} para estimativas atuais.
\href{https://biblioteca.ibge.gov.br/visualizacao/livros/liv101764.pdf}{PDF}

\paragraph{Menezes et al. (2015) -- J. Bras. Pneumol.}
Usando a PNS 2013 para adultos, descreve variações por sexo, idade e região, fornecendo linha de base histórica para comparação temporal com a PNS 2019. \href{https://doi.org/10.1590/1980-5497201500060018}{SciELO}

\paragraph{Ramos, Martins \& Castro (2021) -- Physis.}
Revisão sistemática de estudos regionais, com importante heterogeneidade entre macrorregiões, apoiando a necessidade de modelagem com \emph{survey design} e estratificações/efeitos contextuais. \href{https://ojs.brazilianjournals.com.br/ojs/index.php/BJHR/article/view/30260/pdf}{Link}

% ===================== BLOCO 2 =====================
\subsection{Sono e asma}
\paragraph{Estanislau et al. (2021) -- Jornal de Pediatria.}
No ERICA (adolescentes; $n\approx 59{,}4$ mil), menor duração de sono ($<7$h/noite) associou-se à asma atual (PR $\approx 1{,}17$; IC95\% 1,01--1,35), sugerindo relevância de \emph{sleep hygiene} em faixas jovens.
\href{https://pubmed.ncbi.nlm.nih.gov/32956628/}{PubMed} \,|\, \href{https://doi.org/10.1016/j.jped.2020.07.007}{DOI}

\paragraph{Meltzer et al. (2019) -- Journal of Asthma.}
Associa menor tempo de sono a pior controle/maior exacerbação em adolescentes, reforçando o papel de rotinas e hábitos na expressão clínica.
\href{https://pubmed.ncbi.nlm.nih.gov/31751908}{PubMed}

% ===================== BLOCO 3 =====================
\subsection{Estilo de vida e alimentação}
\paragraph{Lorensia, Suryadinata \& Saputra (2019) -- J. Pharm. Sci. \& Research.}
Estudo comparativo indica níveis mais baixos de vitamina D em asmáticos e papel potencial da atividade física; embora não seja brasileiro, apoia a inclusão de marcadores de dieta/atividade em modelos populacionais.
\href{https://www.researchgate.net/profile/Amelia-Lorensia/publication/333745187_Physical_Activity_and_Vitamin_D_Level_in_Asthma_and_Non-Asthma/links/5d01aec64585157d15a6a73e/Physical-Activity-and-Vitamin-D-Level-in-Asthma-and-Non-Asthma.pdf}{PDF}

% ===================== BLOCO 4 =====================
\subsection{Acesso/uso de serviços e manejo clínico}
\paragraph{Guimarães et al. (2024) -- Braz. J. Implantology \& Health Sciences.}
Síntese prática sobre diagnóstico e tratamento farmacológico/não farmacológico no contexto brasileiro; útil para interpretar variáveis de acesso e manejo (consulta, uso de SUS, medicação).
\href{https://bjihs.emnuvens.com.br/bjihs/article/view/3228}{Artigo online}

\paragraph{Lual \& Awoke (2021) -- Journal of Asthma and Allergy.}
Mostram que baixa adesão, comorbidades e fatores socioeconômicos associam-se a controle subótimo em adultos, conectando determinantes sociais a desfechos de manejo.
\href{https://pubmed.ncbi.nlm.nih.gov/34497674}{Texto completo}

% ===================== BLOCO 5 =====================
\subsection{Ciência de dados e IA na asma}
\paragraph{Finkelstein \& Jeong (2017) -- Ann. NY Acad. Sci.}
Predição precoce de exacerbações via \emph{machine learning}; desempenho promissor em cenários clínicos, indicando potencial para estratificação de risco e alertas.
\href{https://pubmed.ncbi.nlm.nih.gov/27627195}{PubMed}

\paragraph{Kaplan et al. (2021) -- JACI: In Practice.}
Revisão de aplicações de IA/ML em pneumologia (asma/DPOC): oportunidades em imagem, função pulmonar e suporte diagnóstico; ressalta desafios de implementação responsável.
\href{https://pubmed.ncbi.nlm.nih.gov/33618053/}{PubMed} \,|\, \href{https://doi.org/10.1016/j.jaip.2021.02.014}{DOI}

\paragraph{Gonçalves, França \& Zarate (2024) -- Brazilian e-Science Workshop.}
Enfatiza a centralidade do \emph{conhecimento de domínio} para validade de modelos; sustenta a abordagem hierárquica e a integração de especialistas em saúde na análise de inquéritos.
\href{https://www.google.com/url?sa=t&rct=j&q=&esrc=s&source=web&cd=&ved=2ahUKEwjZ8KyJmsCPAxV7LLkGHcMYE9gQFnoECBgQAQ&url=https%3A%2F%2Fsol.sbc.org.br%2Findex.php%2Fbresci%2Farticle%2Fdownload%2F30591%2F30395%2F&usg=AOvVaw26Lp46JpIKoVVGP4wQQEU7&opi=89978449}{SBC (anais)}

% ===================== BLOCO 6 =====================
\subsection{Fenótipo e idade de início}
\paragraph{Hisinger-Molkånen et al. (2022) -- ERJ Open Research.}
Em coortes nórdicas, diagnóstico em idade mais avançada associa-se a sintomas mais graves e pior controle, ressaltando a vantagem do diagnóstico precoce.
\href{https://pubmed.ncbi.nlm.nih.gov/36185544}{PubMed}

% ===================== BLOCO 7 =====================
\subsection{Risco espacial e geoinformação}
\paragraph{Razavi-Termeh, Sadeghi-Niaraki \& Choi (2021) -- ISPRS IJGI.}
Uso de mineração de dados e \emph{ensemble} para mapear áreas urbanas suscetíveis, mostrando utilidade de camadas ambientais/territoriais na vigilância.
\href{https://www.mdpi.com/2072-4292/13/16/3222}{Texto}

% ===================== SÍNTESE CRÍTICA =====================
\subsection{Síntese crítica e lacunas para este estudo}
\begin{itemize}[noitemsep, topsep=0pt]
    \item \textbf{Convergências:} desigualdades sociais e territoriais são recorrentes; comportamentos (tabagismo, atividade física) e marcadores alimentares se associam a pior prognóstico; sono curto agrava desfechos em jovens; manejo depende de acesso/adesão.
    \item \textbf{Oportunidades:} integração de blocos (sociodemográfico, estilo de vida, alimentação, acesso/uso) em um \textit{mesmo} modelo populacional com \emph{survey design} é menos frequente e traz valor para políticas públicas.
    \item \textbf{Gaps:} evidência brasileira que conecte marcadores alimentares \emph{e} acesso/uso de serviços em modelagem hierárquica ainda é limitada; há espaço para análises de sensibilidade com dependências do questionário e imputação explicitamente documentadas.
\end{itemize}

% ----------------------------------------------------------
% REFERÊNCIAS (com links)
% ----------------------------------------------------------
\section{Referências}
\begin{itemize}[leftmargin=0.5cm]
    \item Estanislau, N.R.A. \emph{et al.} (2021). Association between asthma and sleep hours in Brazilian adolescents: ERICA. \textit{Jornal de Pediatria}, 97(4):396--401. \href{https://pubmed.ncbi.nlm.nih.gov/32956628/}{Link} \,|\, \href{https://doi.org/10.1016/j.jped.2020.07.007}{DOI}.
    \item Finkelstein, J.; Jeong, I.C. (2017). Machine learning approaches to personalize early prediction of asthma exacerbations. \textit{Ann NY Acad Sci} 1387(1):153--165. \href{https://pubmed.ncbi.nlm.nih.gov/27627195}{Link}
    \item Gonçalves, L.; França, D.; Zarate, L. (2024). Relevância do entendimento do domínio... \textit{XVIII Brazilian e-Science Workshop}. \href{https://www.google.com/url?sa=t&rct=j&q=&esrc=s&source=web&cd=&ved=2ahUKEwjZ8KyJmsCPAxV7LLkGHcMYE9gQFnoECBgQAQ&url=https%3A%2F%2Fsol.sbc.org.br%2Findex.php%2Fbresci%2Farticle%2Fdownload%2F30591%2F30395%2F&usg=AOvVaw26Lp46JpIKoVVGP4wQQEU7&opi=89978449}{Link}.
    \item Guimarães, A.C.C.M. \emph{et al.} (2024). Diagnóstico e tratamento da asma: uma revisão. \textit{Braz. J. Implantology \& Health Sciences}, 6(8):5230--5240. \href{https://bjihs.emnuvens.com.br/bjihs/article/view/3228}{Link}.
    \item Hisinger-Molkånen, H. \emph{et al.} (2022). Age at diagnosis and disease characteristics of adult-onset asthma. \textit{ERJ Open Research}, 8(1). \href{https://pubmed.ncbi.nlm.nih.gov/36185544}{Link} 
    \item IBGE (2020). PNS 2019: Percepção do estado de saúde... Rio de Janeiro: IBGE. \href{https://biblioteca.ibge.gov.br/visualizacao/livros/liv101764.pdf}{PDF}.
    \item Kaplan, A. \emph{et al.} (2021). AI/ML in respiratory medicine and potential role in asthma/COPD diagnosis. \textit{JACI: In Practice}, 9(6):2255--2261. \href{https://pubmed.ncbi.nlm.nih.gov/33618053/}{Link} \,|\, \href{https://doi.org/10.1016/j.jaip.2021.02.014}{DOI}.
    \item Lorensia, A.; Suryadinata, R.V.; Saputra, A. (2019). Physical activity, vitamin D and asthma. \textit{J Pharm Sci Res}, 11(2):507--510. \href{https://www.researchgate.net/profile/Amelia-Lorensia/publication/333745187_Physical_Activity_and_Vitamin_D_Level_in_Asthma_and_Non-Asthma/links/5d01aec64585157d15a6a73e/Physical-Activity-and-Vitamin-D-Level-in-Asthma-and-Non-Asthma.pdf}{PDF}.
    \item Lual, N.; Awoke, T. (2021). Factors associated with suboptimal asthma control among adults. \textit{Journal of Asthma and Allergy}, 14:1033--1042. \href{https://pubmed.ncbi.nlm.nih.gov/34497674}{Link}.
    \item Meltzer, L.J. \emph{et al.} (2019). The role of sleep in asthma outcomes in adolescents. \textit{Journal of Asthma}, 56(9):957--965. \href{https://pubmed.ncbi.nlm.nih.gov/31751908}{Link}
    \item Menezes, A.M.B. \emph{et al.} (2015). Prevalence of asthma in adults in Brazil: PNS 2013. \textit{J Bras Pneumol}, 41(5):398--404. \href{https://doi.org/10.1590/1980-5497201500060018}{SciELO}.
    \item Ramos, D.; Martins, D.; Castro, M. (2021). Prevalência de asma nas regiões do Brasil: revisão sistemática. \textit{Physis}, 31(2):e310220. \href{https://ojs.brazilianjournals.com.br/ojs/index.php/BJHR/article/view/30260/pdf}{link}.
    \item Razavi-Termeh, S.V.; Sadeghi-Niaraki, A.; Choi, S.-M. (2021). Asthma susceptible area map with data mining. \textit{ISPRS IJGI}, 10(3):178. \href{https://www.mdpi.com/2072-4292/13/16/3222}{Link} 
    \item Rodrigues, L. \emph{et al.} (2021). Asma: uma revisão narrativa. \textit{Revista Brasileira de Medicina}, 78(5):45--53. \href{https://acervomais.com.br/index.php/medico/article/download/9129/5572/}{PDF}.
    \item Xavier, C.C. \emph{et al.} (2022). Seasonal variation and pediatric respiratory admissions. \textit{Rev Paul Pediatr}, 40:e2020186. \href{https://pubmed.ncbi.nlm.nih.gov/37097057}{PubMed}
\end{itemize}

\end{document}
